\documentclass[11pt,a4paper]{article}

%\usepackage[portuguese]{babel}  % if you want Portuguese
\usepackage[utf8]{inputenc}           % 8 bits UTF8
%\usepackage[latin1]{inputenc}     %  OR 8 bits latin1
\usepackage{parskip}            % no indentation on paragraphs
\usepackage{url}                % URLs
\usepackage{lipsum}             % loren dummy text

%% margins
\RequirePackage[outer=25mm,inner=30mm,vmargin=16mm,includehead,includefoot,headheight=15pt]{geometry}

%% headers and footers
\usepackage{fancyhdr}           % page headers
\pagestyle{fancy}
\lhead{}\chead{}\rhead{PD 2022/2023 Sem 1}
%\lfoot{}\cfoot{}\rfoot{Page \thepage}
\renewcommand{\headrulewidth}{0.4pt}
\renewcommand{\footrulewidth}{0.4pt}

\pagenumbering{gobble}

%% some more macros
\newcommand{\dummy}[1]{$<$#1$>$}
\newcommand{\titles}[2]{\noindent\textbf{#1:} #2\\[2mm]}

%% LaTeX exceptions
%\hyphenation{In-fra-struc-ture}

\begin{document}

\titles{Title}{Improving the Developer Experience of Dockerfiles}
\titles{Author}{João Pereira da Silva Matos}
\titles{Supervision}{Filipe Alexandre Pais de Figueiredo Correia}
\titles{Date}{\today}

\section*{Abstract}

Nowadays, containerization is a technique used in a very large number of systems to address problems associated with deployment. The most popular tool used to perform this task is Docker. In order to use Docker, a developer must create a Dockerfile, a configuration file that is used to create the containers. Creating these files can be difficult \cite{reisLiveDockerContainers2020} and even functional files can have problems. In fact (add a reference about how a large percentage of dockerfiles are vulnerable)...




\titles{Keywords}{Dockerfile, Docker, File generation, File repair}
\titles{ACM Classification}{CCS - Software and its engineering - Software notation and tools - Software configuration management and version control systems
}



\nocite{*}  % to include references which were not cited

%% the references using BibTeX
\bibliographystyle{unsrt}
\bibliography{abstractBib}

\end{document}
